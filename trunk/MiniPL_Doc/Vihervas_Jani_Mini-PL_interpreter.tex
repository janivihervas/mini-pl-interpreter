\documentclass[a4paper,12pt]{article}
\usepackage[utf8]{inputenc}
\usepackage[T1]{fontenc}
\usepackage[english]{babel}
\usepackage[charter]{mathdesign}
\usepackage{beton}
\renewcommand{\bfdefault}{bx}
\renewcommand{\scdefault}{sc}
\usepackage{textcomp}
\usepackage{hyperref}
\usepackage{enumerate}
\usepackage{amsmath}
\usepackage[all]{xy}
\usepackage[left=1.5cm,right=1.5cm,top=1.5cm,bottom=1.5cm]{geometry}
\setlength{\parindent}{0pt}
\setlength{\parskip}{1em}

\author{Jani Viherväs}
\hypersetup{pdfinfo={
Title={Mini-PL Interpreter},
Author={Jani Viherväs},
Subject={Compilers},
Keywords={compilers, Mini-PL, interpreter}
}}


\newcommand{\ii}[1]{\textit{#1}}
\newcommand{\bb}[1]{\textbf{#1}}
\newcommand{\ttt}[1]{\texttt{#1}}
\newcommand{\e}{\epsilon}
\newcommand{\s}{\ttt{ }}
\newcommand{\integer}{\ttt{i}}
\newcommand{\str}{\ttt{s}}
\newcommand{\bool}{\ttt{b}}


\begin{document}

\begin{flushright}
\today \\
\vspace{1em}
Jani Viherväs\\ 
\href{mailto:jani.vihervas@cs.helsinki.fi}{jani.vihervas@cs.helsinki.fi}
\end{flushright}

\vfill

\begin{center}
\textsc{\LARGE Mini-PL Interpreter} \\
\vspace{1em}
\textsc{\large 58144 Compilers Project}
\end{center}

\vfill

Grammar:
\begin{align*}
<prog>\s \to \s &<stmts> \\
<stmts>\s \to \s &<stmt> \bb{ ; } <stmts'> \\
<stmts'>\s \to \s &\e \mid <stmts> \\
<stmt>\s \to \s &\bb{var } <ident'> \bb{ : } <type> <stmt'> \\ 
       \mid \s &<ident> \bb{ := } <expr> \\  
       \mid \s &\bb{for } <ident> \bb{ in } <expr> \bb{ .. } <expr> \bb{ do } 
             <stmts> \bb{ end } \bb{for} \\
       \mid \s &\bb{read } <ident> \\
       \mid \s &\bb{print } <expr> \\
       \mid \s &\bb{assert (}  <expr> \bb{ )} \\
<stmt'>\s \to \s &\e \mid \bb{:= } <expr> \\
<expr>\s \to \s &<opnd> <op> <opnd> \\
<expr'>\s \to \s &\e \mid <unary> \\
<opnd>\s \to \s &<int> \\
       \mid \s &<string> \\
       \mid \s &<expr'><bool> \\
       \mid \s &<ident> \\
       \mid \s &\bb{( } <expr> \bb{ )} \\
<type>\s \to \s &\bb{int} \mid \bb{string} \mid \bb{bool} \\
<reserved\s keyword>\s \to \s 
              &\bb{var} \mid \bb{for} \mid \bb{end} \mid \bb{in} \mid \bb{do}
              \mid \bb{read} \mid \bb{print} \mid \bb{assert} \\ 
              \mid \s &\bb{int} \mid \bb{string} \mid \bb{bool}  \mid \ttt{true} \mid \ttt{false}\\
<unary>\s \to \s &\bb{!} \\
<op>\s \to \s &\bb{+} \mid \bb{-} \mid \bb{*} \mid \bb{/} \mid \bb{<} \mid
\bb{>} \mid \bb{<=} \mid \bb{>=} \mid \bb{=} \mid \bb{\&}
\end{align*}

$<ident'>$ adds identifier to symbol table, where as $<ident>$ looks the
identifier from the symbol table. Operators $>$, $<=$ and $>=$ are added,
because they are very easy to implement.

Predict sets:
\[
\begin{array}{r|c}
\text{Production} & \text{Predict set} \\
\hline
<prog>   & \bb{var}, <ident>, \bb{for}, \bb{read}, \bb{print}, \bb{assert} \\
\hline
<stmts>  & \bb{var}, <ident>, \bb{for}, \bb{read}, \bb{print}, \bb{assert} \\
\hline
<stmts'> & \$\$ \\
         & \bb{var}, <ident>, \bb{for}, \bb{read}, \bb{print}, \bb{assert} \\
\hline
<stmt>   & \bb{var}  \\
         & <ident>  \\
         & \bb{for}  \\
         & \bb{read}  \\
         & \bb{print}  \\
         & \bb{assert}  \\
\hline
<stmt'>  & \bb{;} \\
         & \bb{:=}  \\
\hline
<expr>   & \integer, \str, \bb{!}, \bool, <ident>, ( \\
\hline
<expr'>  & \bool \\
         & \bb{!}  \\
\hline
<opnd>   & \integer  \\
         & \str  \\
         & \bb{!}, \bool  \\
         & <ident>  \\
         & (  \\
\hline
<type>   & \bb{int} \\
         & \bb{string} \\
         & \bb{bool} \\
\hline
\end{array}
\]

\end{document}
